\documentclass[12pt]{article}

%\documentclass[twoside]{article}
%\usepackage{a4}
\usepackage[letterpaper,top=0in, left=0in, bottom=1in, right=0in]{geometry} %So margins are not tsheir default huge
\usepackage{amssymb}
\usepackage{bigints}
\usepackage{amsmath}
\usepackage[title]{appendix}
\usepackage{graphicx}
\usepackage{booktabs,caption}
\usepackage[flushleft]{threeparttable}
\usepackage{caption}
\usepackage{subcaption}
\usepackage{float}
\usepackage{upref}
\usepackage[active]{srcltx}
\usepackage[sort, comma]{natbib}
\usepackage[colorlinks,citecolor=blue,linkcolor=blue]{hyperref}
\usepackage[dvipsnames]{color}
\usepackage[english]{babel}
\usepackage[utf8]{inputenc}
\usepackage{fancyhdr}
\allowdisplaybreaks[2] % To make it possible for displaybreaks
%
% Color commands
%
% Note that white is shown as black by xdvi
%

%
\definecolor{blau}{rgb}{0.1,0.0,0.9}
\definecolor{purple}{rgb}{0.4,0.0,0.9} % By Anders Bj\"orn 2012-09-26
\definecolor{gruen}{cmyk}{1.0,0.2,0.7,0.07}
\definecolor{mag}{cmyk}{0.0,0.9,0.3,0.0}
\newcommand{\blue}{\color{blau}}
\newcommand{\purple}{\color{purple}}
\newcommand{\green}{\color{gruen}}
\newcommand{\magenta}{\color{mag}}

\usepackage{graphicx}
\usepackage{booktabs}
\usepackage{geometry}
\usepackage{titling}
\usepackage{lipsum}
\newgeometry{top=1in,bottom=1in,outer=1in,inner=1in}

\parskip = 3pt

%\newcommand{\blue}{\color{blue}}
\newcommand{\red}{\color{red}}
%\newcommand{\green}{\color{green}}
\newcommand{\yellow}{\color{yellow}}
\newcommand{\cyan}{\color{cyan}}
%\newcommand{\magenta}{\color{magenta}}
\newcommand{\white}{\color{white}}
\newcommand{\black}{\color{black}}
%
\newcommand{\authortitle}[2]{\author{#1}\title{#2}\markboth{#1}{#2}}
%
%       Reference definitions
%%
%       Theorems and other things numbered
%
\renewcommand{\baselinestretch}{1.5} 


\newtheorem{thm}{Theorem}[section]
\newtheorem{theo}[thm]{Theorem}   % Alternative
\newtheorem{prop}[thm]{Proposition}
%\newtheorem{prop-ten}[thm]{Proposition (tentative)}
%\newtheorem{thm-ten}[thm]{Theorem (tentative)}
\newtheorem{lem}[thm]{Lemma}
\newtheorem{cor}[thm]{Corollary}
\newtheorem{theorem}[thm]{Theorem}
\newtheorem{lemma}[thm]{Lemma}
\newtheorem{proposition}[thm]{Proposition}
\newtheorem{corollary}[thm]{Corollary}
\newtheorem{claim}[thm]{Claim}
\newtheorem{deff}[thm]{Definition}
\newtheorem{definition}[thm]{Definition}
\newtheorem{defi}[thm]{Definition} % Alternative
\newtheorem{example}[thm]{Example}
%\newtheorem{conj}[thm]{Conjecture}
\newtheorem{prob}[thm]{Problem}
%\theoremstyle{remarkstyle}
% * may be inserted for nonnumbering
\newtheorem{remark}[thm]{Remark}
\newtheorem{rem}[thm]{Remark} % Alternative
\newtheorem{openprob}[thm]{Open problem}

%
% Here we make the equation numbering to be within sections
%
%
%  \cprime used for the Russian soft sign 
%


\newcommand{\vp}{\varphi}
%\renewcommand{\phi}{\varphi}
\newcommand{\eps}{\varepsilon}
\newcommand{\lm}{\lambda}
\newcommand{\al}{\alpha}
\newcommand{\be}{\beta}
\newcommand{\de}{\delta}
\newcommand{\E}{\mathbb E}
\newcommand{\tTheta}{{\widetilde{\Theta}}}
\renewcommand{\l}{\left}
\renewcommand{\r}{\right}

\newcommand{\R}{\mathbf{R}}
\newcommand{\Q}{\mathbf{Q}}
\newcommand{\Qp}{{\Q_\limplus}}
\newcommand{\Rn}{\mathbf{R}^n}
\newcommand{\Rno}{\mathbf{R}^{n+1}}
\newcommand{\B}{{\mathcal B}}

\newcommand{\dist}{\operatorname{dist}}

\renewcommand{\limsup}{\operatornamewithlimits{lim\, sup}}
\renewcommand{\liminf}{\operatornamewithlimits{lim\, inf}}
\newcommand{\esssup}{\operatornamewithlimits{ess\, sup}}
\newcommand{\essinf}{\operatornamewithlimits{ess\,inf}}
 \DeclareMathOperator*{\essliminf}{ess\,lim\,inf}
 \DeclareMathOperator*{\esslimsup}{ess\,lim\,sup}

%\newcommand{\supp}{\operatorname{spt}}
\newcommand{\spt}{\operatorname{spt}}


\newcommand{\divt}{\operatorname{div}}
\renewcommand{\div}{\nabla \cdot}


\newcommand{\parts}[2]{\frac{\partial {#1}}{\partial {#2}}}
\newcommand{\pdo}[2]{\frac{\partial #1}{\partial #2}}

\newcommand{\Qed}[0]{\nopagebreak[4]\hspace{\stretch{3}}$\Box$\vspace{11 pt}}
\newcommand{\abs}[1]{\left| #1 \right|}

\newcommand{\wl}{w_\lambda}
\newcommand{\ol}{\overline}

\newcommand{\theoref}[1]{Theorem~\ref{#1}}
\newcommand{\propref}[1]{Proposition~\ref{#1}}
\newcommand{\lemref}[1]{Lemma~\ref{#1}}
\newcommand{\remref}[1]{Remark~\ref{#1}}
\newcommand{\defiref}[1]{Definition~\ref{#1}}
\newcommand{\secref}[1]{Section~\ref{#1}}
\newcommand{\corref}[1]{Corollary~\ref{#1}}

\newcommand{\trm}{\textrm}

\newcommand{\half}{{\frac{1}{2}}}

\newcommand{\Om}{\Omega}
\newcommand{\Th}{\Theta}

% \p for better spacing in constructions like p-something
\newcommand{\p}{{$p\mspace{1mu}$}}


\newcommand{\uS}{\itoverline{S}}
\newcommand{\lS}{\itunderline{S}}
\newcommand{\uP}{\itoverline{P}}
\newcommand{\uR}{\itoverline{R}}
\newcommand{\uPa}{{\itoverline{P}\mspace{1mu}}^a}
\newcommand{\lP}{\itunderline{P}}
\newcommand{\lR}{\itunderline{R}}
\newcommand{\uPind}[1]{\itoverline{P}_{#1}}
\newcommand{\lPind}[1]{\itunderline{P}_{#1}}
\newcommand{\UU}{\mathcal{U}}%
\newcommand{\UUt}{\widetilde{\mathcal{U}}}%
\newcommand{\LL}{\mathcal{L}}%
\newcommand{\LLt}{\widetilde{\mathcal{L}}}%
\newcommand{\bdy}{\partial}
\newcommand{\bdry}{\partial}
\newcommand{\bdyp}{\bdy_p}
\newcommand{\grad}{\nabla}
\newcommand{\setm}{\setminus}
\renewcommand{\emptyset}{\varnothing}
 \newcommand{\alp}{\alpha}
 \newcommand{\ga}{\gamma}
\newcommand{\tf}{\tilde{f}}
\newcommand{\wcj}{w_{c,j}}
\newcommand{\ucj}{u_{c,j}}
\newcommand{\vcj}{v_{c,j}}
\newcommand{\dcj}{d_{c,j}}
\newcommand{\etacj}{\eta_{c,j}}
\newcommand{\psicj}{\psi_{c,j}}
\newcommand{\Hcj}{H_{c,j}}

\newcommand{\la}{\lambda}
\newcommand{\wt}{\widetilde{w}}
\newcommand{\Kt}{\widetilde{K}}
\newcommand{\Gt}{\widetilde{G}}
\newcommand{\Wp}{W^{1,p}}
\newcommand{\Omm}{\Om_\limminus}
\newcommand{\efoft}{\biggl(\frac{|{\log(-t)}|^{p-2}-1}{p-2}\biggr)}
\newcommand{\efoftsing}{\biggl(\frac{|{\log(-t)}|^{2-p}-1}{2-p}\biggr)}

\newcommand{\loc}{_{\rm loc}}
\newcommand{\Lploc}{L^{p}\loc}
\newcommand{\ut}{\tilde{u}}
\newcommand{\ft}{\tilde{f}}
\newcommand{\Thetam}{\Theta_\limminus}

\setcounter{secnumdepth}{3}

%\thispagestyle{plain}
\pagestyle{fancy}
\fancyhf{}
\rhead{Econ PS1}
\chead{}
\lhead{Suleyman Gozen}
\cfoot{\thepage}


\begin{document}
\textbf{Exercise 1}

1.  We just need to know how many remaining barrels of oil the owner of an oil field has in each period $t$. Let's denote it by $x_{t}$ where $\sum_{t=0}^{\infty} x_{t} = B$.

2. The control variables are the amount of barrels of oil the owner of an oil field chooses to sell in each period $t$. Let's denote it by $y_{t} \in [0,B]$.

3. The transition equation can be summarized as $x_{t+1} = x_{t} - y_{t}$ for each $t \geq 0$ as long as $x_{t+1} \geq 0$.

4. \textbf{Sequence problem}:
\begin{align*}
\max_{0 \leq y_{t} \leq B} \sum_{t=0}^{\infty} & {(\frac{1}{1+r})}^{t}p_{t}y_{t}  \\ & \text{s.t.} \quad  x_{t+1} = x_{t} - y_{t} \\ & \quad \quad \sum_{t=0}^{\infty} x_{t} = B  \\ & \quad \quad x_{t} \geq 0 \quad \text{for all t}\geq 0
\end{align*}


\textbf{Bellman equation}:
\begin{align*}
V(x_{t}) = & \max_{0 \leq y_{t} \leq B}  p_{t}y_{t} + \frac{1}{1+r} V(x_{t+1})   \\  & \text{s.t.} \quad  x_{t+1} = x_{t} - y_{t} \\ & \quad \quad \sum_{t=0}^{\infty} x_{t}= B  \\ & \quad \quad x_{t} \geq 0 \quad \text{for all t}\geq 0
\end{align*}

5. Euler equation looks like a bang-bang situation, i.e. 
\begin{itemize}
\item for $y_{t} \in (0,B)$, $p_{t} = p_{t+1}\frac{1}{1+r}$ for all $t \geq 0$.
\item for $y_{t} = 0$, $p_{t} < p_{t+1}\frac{1}{1+r}$ for all $t \geq 0$.
\item for $y_{t} = x_{t}$ , $p_{t} > p_{t+1}\frac{1}{1+r}$ for all $t \geq 0$.
\end{itemize}

\clearpage

6.  \begin{itemize} \item If $p_{t+1} = p_{t}$ for all $t$, then there is no motivation for an owner to wait and hence the optimal solution is selling the all at the initial period, i.e.  $y_{0} = B$ and $y_{t} = 0$ for all $t >0$. 
\item If $p_{t+1} > p_{t}(1+r)$ for all $t$, then the best strategy for an owner is waiting forever. If there is no terminal condition, there would not exist an equilibrium in this case. 

Therefore, for an interior solution, the path of prices should be $p_{t+1}\frac{1}{1+r} = p_{t}$ for all $t$.

\end{itemize}

\textbf{Exercise 2}

1. The state variable is $k_t$.

2. The control variables are $\{c_t, k_{t+1}\}$.

3. \textbf{Bellman equation}
\begin{align*}
V(k_{t}) = & \max_{c_{t} \in [0,z_{t}k_{t}^{\alpha}]}  u(c_{t}) + \beta \mathbb{E}_{t} [V(k_{t+1})]  \\  & \text{s.t.} \quad  c_{t} + k_{t+1} = z_{t}k_{t}^{\alpha} + (1-\delta) k_{t} \\  & \quad \quad \quad k_{t} \geq 0 \quad \text{for all t}\geq 0 \\ & \quad \quad \quad ln(z_{t}) \sim N(0, \sigma_{z})
\end{align*}

where $ \mathbb{E}_{t} = \int z_{t} \phi (z_{t})dt$ and $\phi(.)$ is pdf of normal distribution. 

4. You can find the answers in my "$EconPS1.ipynb$" file under Exercise 2 section.

\textbf{Exercise 3}

1. \textbf{Bellman equation}
\begin{align*}
V(k_{t}) = & \max_{c_{t} \in [0,z_{t}k_{t}^{\alpha}]}  u(c_{t}) + \beta \mathbb{E}_{t} [V(k_{t+1})|z_{t}]   \\  & \text{s.t.} \quad  c_{t} + k_{t+1} = z_{t}k_{t}^{\alpha} + (1-\delta) k_{t} \\  & \quad \quad \quad k_{t} \geq 0 \quad \text{for all t}\geq 0 \\ & \quad \quad \quad ln(z_{t}) = \rho ln(z_{t-1}) + \upsilon_{t}
 \end{align*}

2. You can find the answers in my "$EconPS1.ipynb$" file under Exercise 3 section.


\textbf{Exercise 4}

1. \textbf{Bellman equation}

\begin{align*}
V(w_{t}) = & \max \{\frac{w_{t}}{1-\beta}, b + \beta \int_{0}^{W} V(\tilde{w})f(\tilde{w})d\tilde{w}\}  
 \end{align*}
 
 where $f(.)$ is density of the wage distribution with cdf $F$. Here, I assume $F(b) < 1$ (at least some offers are higher than b) and $F(W) = 1$ for some $W$. 

2. \textbf{Reservation wage}:

$$V(w_{R}) = \frac{w_{R}}{1-\beta} = b + \beta [F(w_{R})V(w_{R}) + \int_{w_{R}}^{W} \frac{\tilde{w}}{1-\beta}f(\tilde{w})d\tilde{w}$$

If we subtract $\frac{\beta}{1-\beta}w_{R}$, we get

$$w_{R} - b = \frac{\beta}{1-\beta} [\int_{w_{R}}^{W} \frac{\tilde{w}}{1-\beta}f(\tilde{w})d\tilde{w}]$$

You can find the other answers in my "$EconPS1.ipynb$" file under Exercise 4 section.
 


















\end{document}

